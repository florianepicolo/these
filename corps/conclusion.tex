\chapter{Conclusion}
\thispagestyle{firstpage}
\onehalfspacing

\par Nos travaux ont testé trois scénarios possibles de construction des voies de signalisation. Nous mettons également en lumière deux moments d’apparition clés dans l’arbre de la vie pour les gènes impliqués dans une voie de signalisation qui sont le nœud des Opisthoconte et celui des Vertébrés. Au regard de ces résultats, la double duplication de génomes à l’émergence des Vertébrés peut être à l’origine de la naissance de nouveaux gènes de voie de signalisation. Nous avons poussé nos recherches en détail pour connaître le comportement des interactions chez des clades les plus anciens des vertébrés, à savoir les téléostéens qui ont subi 3 ou 4 duplications de génome. Nous avons constaté que les orthologues téléostéens des gènes humains des voies de signalisation restaient plus souvent en duplicat ou en triplicat chez les espèces 3 ou 4 WGD téléostéens que les gènes du génome. Et nous avons retrouvé une majorité d’interactions protéine-protéine à stœchiométrie respectée en moyenne chez les téléostéens.