\chapter{Discussion}
\thispagestyle{firstpage}
\onehalfspacing

\section{Étude sur les moments d’apparition}
\par Dans un premier temps, notre étude a permis de mettre en lumière trois scénarios possibles sur la construction d’une voie de signalisation intracellulaire humaine. Notre premier scénario était que les voies sont arrivées dans le sens amont vers aval, donc des ligand-récepteur aux facteurs de transcription. Cette hypothèse nous semblait être la moins probable au vu des résultats de l’étude d’Anna Grandchamp durant sa thèse \parencite{Grandchamp & Monget, 2018}. En effet, ce travail a montré que 38\% des récepteurs sont apparus avant leur ligand contre 21\%, apparus après leur ligand. Or dans notre cas, le ligand est à la position 1 dans la cascade de signalisation, et le récepteur à la position 2 ce qui favoriserait la deuxième hypothèse. Cependant, nous avons travaillé sur un plus grand nombre d’espèces : 315 au lieu des 145 espèces dans l’étude précédente, et sur 25 clades au lieu de 10. Cette augmentation des données permet probablement d’être plus précis dans la recherche d’orthologues dans l’arbre de la vie, et d’être plus résolutifs concernant les nœuds d’apparition. Ce scénario pourrait expliquer 10 des voies KEGG. 
\par Notre deuxième scénario concerne les voies qui se sont construites de l’aval vers l’amont, donc des facteurs de transcription vers les ligand-récepteur et a été validé pour 25 voies de notre étude (r < 0, p value < 0.02) dont 3 voies (Adipocytokine, Fc epsilon RI et VEGF) pour lesquelles la corrélation est particulièrement forte (r > 0.5, p value < 3.15E-8). Ce scénario peut également être motivé par d’autres éléments, comme le fait que les protéines en aval de voie de signalisation sont connectées avec plus de partenaires que les protéines en amont de la voie. La “multi-connectivité“ d’une protéine augmente ses contraintes évolutives car cela imposerait une co-évolution à ses différents partenaires \parencite{fraser_evolutionary_2002, hahn_molecular_2004, krylov_gene_2003}. 
\par Pour autant, cela pourrait également venir des types de fonction en amont des voies et en aval qui peuvent être différents. En effet, dans l’étude d’\cite{alvarez-ponce_relationship_2012}, il est montré une différence notable entre la nature des protéines en début et en fin de voie. Les protéines en amont d’une voie sont logiquement enrichies en activité kinases (q = 2,24 × 10 –4, q étant la p-value corrigée par Benjamini et Hochberg) tandis que les protéines en aval sont enrichies en activités de transporteur transmembranaire ionique (q = 2,65 × 10 –5 ). De notre côté, nous n’avons pas regardé la \textit{Gene Ontology} de nos protéines en début et en fin de voie, mais cela pourrait être reconsidéré.
\par Et notre dernier scénario concernait une absence d’ordre pour la construction d’une voie de signalisation. Ce scénario semble concerner 12 voies de notre étude. 
\par Par ailleurs, l’étude d’Alvarez-Ponce qui comprenait 1049 protéines et 2436 interactions, a montré qu’il n’y avait pas d’incidence entre la position hiérarchique d’une protéine dans une voie et leur taux d’évolution. La particularité de l’étude était qu’il partait sans a priori de la voie c’est-à-dire qu’ils ont utilisés uniquement des interactions et non pas des voies définies comme le fait KEGG et la position hiérarchique était déterminée en fonction de leur position dans la voie (\textit{upstream} ou \textit{downstream}). Le taux d’évolution représente le ratio entre les mutations silencieuses et les mutations faux sens pour des séquences hommes vs. souris \parencite{alvarez-ponce_relationship_2012}. Dans notre cas, le taux d’évolution (dN/dS) n’a pas été calculé. Nous pourrions également envisager de le faire pour notre étude.
\par Le taux d’évolution est donc différent du moment d’apparition, mais peut lui être lié. Plus un gène a un taux d’évolution élevé, par exemple s’il est soumis à une sélection positive, et plus il y a de chance qu’il ait beaucoup divergé jusqu’à devenir une sorte de nouveau gène et donc de ne pas être détecté en tant qu’orthologue. Dans ce cas, le moment d’apparition du gène humain pourrait d’être plus éloigné que pour un gène présentant une vitesse d’évolution plus lente. 
\par Nous avons fait le choix durant cette étude de partir de l’homme, qui est l’espèce la plus référencée, et de partir d’un échantillon de voies de signalisation décrite comme telle sur la base de données KEGG. 
\par De plus, nous avons imaginé 3 scénarios possibles, mais il existe d’autres scénarios auxquels nous n’avons exploré. On peut évoquer maintenant que des voies pourraient être construites à partir d’une chaîne d’interactions protéiques ancestrales « simple » et se serait vu complexifiée à travers l’évolution et les différentes spécificités des espèces. Cette réflexion s’est notamment faite en analysant de plus près la sous-voie RTK/RAS/ERK qui est présente dans 22 voies sur 47 de notre étude. Les éléments de cette sous-voie sont apparus tôt dans l’arbre de la vie puisque la voie est commune aux drosophiles, aux vers et aux humains \parencite{ashton-beaucage_signalisation_2010}. 
\par Cette première étude a également mis en lumière deux nœuds qui semblent être des pivots de l’évolution pour les gènes de voie de signalisation : le nœud des Opisthoconte, et celui des Vertébrés. Ce sont ces mêmes nœuds qu’Anna Grandchamp avait mis en évidence dans son étude sur les ligand-récepteur \parencite{grandchamp_synchronous_2018}. Le nœud des Opisthoconte est un nœud peu surprenant au vu de la littérature. On constate que la comparaison levures-mammifères a montré de grandes similitudes de structure de réseaux moléculaires (Cross et al., 2011). Concernant le deuxième nœud évolutif notable, le nœud des Vertébrés, il vient prendre son origine après deux duplications de génome, et il a été montré également que les gènes issus de duplication de génome avaient une évolution lente comparée aux gènes dupliqués à petite échelle \parencite{satake_evolution_2012}. Dans cette même étude également, sont pointés du doigt les gènes d’expression ubiquiste qui évolueraient moins vite que les gènes d’expression tissulaires spécifiques. Et c’est effectivement un point qui pose une autre question : Est-ce que les gènes impliqués dans des voies de signalisation activées dans la majorité des cellules évoluent moins vites que des voies spécifiques de tissus ? Effectivement, certaines voies sont essentielles à toutes les cellules, notamment les voies menant à la croissance cellulaire ou la division cellulaire comme la voie cAMP \parencite{sassone-corsi_cyclic_2012} et d’autres spécifiques à une fonction comme les voies de l’immunité ou la voie Hippo régulatrice de la contraction musculaire entre autres grâce à sa capacité régulatrice de la taille des cellules \parencite{zhao_hippo_2011}. 
\par Un début de réponse peut être donné concernant les gènes des voies de l’immunité puisque nous avons trouvé que 72\% des gènes des voies de l’immunité sont apparus après le nœud des Vertébrés. Ce constat rejoint par ailleurs la littérature qui évoque l’évolution rapide de ces gènes \parencite{cooper_evolution_2006, schlesinger_coevolutionary_2014}. 
\par Il aurait été intéressant de confronter différentes voies de signalisation présentes chez des espèces différentes de l’arbre de la vie des Opisthocontes. Seulement, la voie MAPK est l’unique voie est disponible chez la levure sur KEGG.
\par Il faut tout de même émettre des réserves sur nos résultats car un moment d’apparition pour une interaction ne signifie pas qu’elle est fonctionnelle. Tout comme un moment d’apparition d’un gène ne rend pas la protéine fonctionnelle chez toutes les espèces concernées. Il faudrait, pour confirmer ces résultats, valider fonctionnellement les interactions chez les espèces avec l’ancêtre commun le plus éloigné de l’homme par un système de double hybride par exemple. Un système de double hybride permettrait de valider une interaction entre deux protéines, mais dans des conditions artéfactuelles, pouvant être très éloignée de la réalité. Ce qui serait un travail colossal et très coûteux pour le nombre d’interactions et le nombre d’espèces différentes. On pourrait également utiliser des techniques novatrices et relativement inexplorées comme la résurrection des gènes. Des travaux ont été entamés sur la résurrection de protéines afin de mieux connaître les fonctions des gènes ancestraux. La technique consiste à réintégrer dans un organisme un gène jusqu’alors inactif ou désactivé chez cette espèce par le biais d’un système tel que Crispr-Cas9. Bien que ce soit une technique complexe, elle se veut prometteuse \parencite{harms_evolutionary_2013}. Une autre façon de répondre à ces questions serait de modéliser les structures tridimensionnelles des protéines ancestrales. La technologie AlphaFold2 est une avancée majeure dans le domaine de la prédiction de la structure des protéines \parencite{cramer_alphafold2_2021}. Avec cet outil, nous pourrions pousser l’étude d’un point de vue tridimensionnel et se demander pour chacune des protéines des voies de signalisation, et à travers l’arbre de la vie, à quel moment ont eu lieu les modifications tridimensionnelles, si changement il y a. Et pourquoi ne pas rêver d'arbres phylogénétiques de gènes basés non pas sur des homologies de séquences primaires, mais sur des homologies de structure 3D des protéines ?
\par Les gènes et leur évolution sont des sujets qui animent la communauté scientifique. Par ailleurs, l’évolution est constante et perpétuelle, ce qui mène à des questionnements peut être plus hardis. Par exemple, on pourrait se poser la question de futures duplications de génome chez certains clades. Certaines espèces vivant dans des régions arides ont peut-être déjà doublé certains de leurs gènes pour mieux supporter la chaleur. Et si les températures continuent d'augmenter, il est possible que, sur plusieurs siècles, les poissons développent des duplications génétiques pour accroître leur résistance à la chaleur. De même, ils pourraient évoluer pour mieux s'adapter à des environnements plus salins si les rivières venaient à s'assécher complètement. Grâce au système de Crispr-Cas9, nous arrivons de mieux en mieux à modifier les génomes précisément. Il pourrait donc être envisageable de dupliquer des génomes. Mais ceci reste encore très utopique. 
\par Dans un registre beaucoup plus réalisable, il serait intéressant de regarder plus en profondeur les gènes orphelins, car l’étude a montré qu’une majorité protéines en interactions attendaient leur partenaire (84\% d’interactions asynchrones). Que font les protéines sans leur partenaire ? Premièrement, nous avons choisi de centrer notre étude sur les gènes des voies de signalisation qui représentent 3 000 interactions protéine-protéine unique, ce qui en fait un petit échantillon sur les 93 000 interactions protéine-protéine uniques recensées par \cite{luck_proteome-scale_2017}. Et deuxièmement, des études ont été menées sur des récepteurs nucléaires orphelins, mais les résultats sont trop divergents, ce qui montre bien la difficulté de la problématique \parencite{markov_origin_2011}. 

\newpage
\section{Étude sur les téléostéens}
\par Dans un deuxième temps, nous avons étudié la stœchiométrie des interactions chez les espèces 3 WGD et 4 WGD du clade des téléostéens. Cette étude a montré que pour l’ensemble des espèces de poissons téléostéens, la stœchiométrie des interactions était respectée en fonction du nombre de duplications de génome des espèces. Pour autant, une notion n’a pas été prise en compte : les paralogues retrouvés sont-ils des paralogues ohnologues, ou des paralogues « simples ». On recense environ 26\% de gènes ohnologues chez le poisson zèbre \parencite{howe_zebrafish_2013} et environ 44\% de gènes ohnologues chez les salmonidés \parencite{dimos_homology_2023}.
\par Ce que nous avons trouvé étonnant, ce sont les résultats concernant les espèces 4 WGD. Pour rappel, les clades des salmonidés et des carpes ont vécu indépendamment une 4ème duplication de génome, à des temps différents et pour autant, les résultats concernant les différentes stœchiométries possibles pour ces deux groupes d’espèces étaient sensiblement les mêmes. La 4ème duplication de génome des Salmonidés a eu lieu, il y a environ 80 millions d’années \parencite{lien_atlantic_2016}, tandis que celle des carpes a eu lieu il y a environ 14 millions d’années \parencite{jaillon_genome_2004, kon_single-cell_2022, xu_allotetraploid_2019}. Nous avons constaté des proportions similaires concernant les différentes stœchiométries, mais nous n’avons pas étudié si les interactions en question étaient les mêmes pour les deux clades indépendants. Nous pourrions lister ces interactions afin de mieux comprendre si la pression de maintien des interactions en stœchiométrie respectées étaient les mêmes malgré une divergence aussi lointaine. 
\par De plus, il serait intéressant d’avoir une approche par voie pour les téléostéens, c’est-à-dire regarder si des sous-voies au sein des voies sont en stœchiométrie respectée. Dans notre cas, nous avons étudié interaction par interaction et non directement en utilisant la voie. C’était dû à un manque de maîtrise des outils igraph au moment de l’étude, mais ça pourrait mettre en lumière une pression de maintien en duplicat ou même de retour en singleton par la voie au sein des téléostéens. Ça pourrait donner également des indications sur les voies de signalisation existantes chez les téléostéens qui sont encore assez peu étudiées. 
\par Nous avons également noté une part non négligeable de gènes absents chez les téléostéens. Les interactions partiellement ou totalement perdues chez les téléostéens étaient en moyenne à hauteur de 25\%. Dans quelques cas, notamment dans le cas où nous avions une perte chez quelques espèces tandis que l’interaction était présente chez le reste des espèces, nous avons retrouvé des traces de pseudogènes. Mais, dans le cas où nous ne retrouvions pas l’interaction pour l’ensemble du clade, nous avons émis l’hypothèse que les gènes et donc les interactions seraient nées après. Seulement, il serait pertinent de croiser nos deux études et de regarder si les gènes étaient apparus avant le clade des Vertébrés, mais se seraient éteints chez les téléostéens car non essentiel aux espèces. De plus, cette hypothèse est un biais de notre étude que nous pourrions améliorer en redéfinissant les conditions d’admission d’un nœud d’apparition. Nous pourrions convenir qu’il faut plus de 80\% (par exemple) de présence du gène dans l’ensemble des espèces entre l’homme et le nœud le plus éloigné où on retrouve un orthologue.  


\newpage
\section{Limite du matériel et de la méthode}
\par Durant ces différentes études, nous avons fait des choix de matériel à utiliser et d’une méthodologie à suivre. Nous avons longuement hésité entre une étude exhaustive avec toutes les voies de signalisation que proposait KEGG, et même pousser l’exhaustivité et regrouper des données de différentes bases de données comme celles de la base de données BioGrid, String, Reactome ou encore NetPath qui sont également des bases de données d’interactions protéiques, ou faire une étude plus spécifique sur une voie de signalisation précise qui pourrait notamment intéresser l’équipe de recherche dans laquelle j’ai réalisé ma thèse. Nous avons fini par trancher sur une liste de 47 voies de signalisation sur la base de données KEGG, car c’était la mieux référencée \parencite{rigden_26th_2019}. 
\par Concernant l’arbre et les espèces utilisées, Anna Grandchamp avait utilisé ces deux mêmes bases de données. Cette thèse s’inscrivant naturellement à la suite de celle d’Anna Grandchamp en 2018, il y avait un souhait de confronter nos résultats aux siens et donc d’utiliser les mêmes outils. Pour autant, la base de données Ensembl et par extension Genomicus a connu un changement important entre la version 93 (janvier 2018) et la version 94 (juillet 2018). En version 93, les arbres étaient présentés de telle façon à ce que l’ensemble des paralogues et orthologues d’un gène soient présents dans un arbre unique. Seulement, avec les ajouts de nouvelles espèces séquencées, la base de données a changé de méthodes de clustering des gènes et de nombreux liens de paralogie ont disparu, rendant impossible la visualisation des grandes familles de gènes. Les arbres ont donc été redécoupés en sous-arbre \parencite{emily_changes_2018}. De ces changements majeurs, nous nous sommes longuement posé la question de savoir si nous devions rester sur les arbres de version 93, ou si nous prenions les versions les plus récentes. Pour nos questions de recherche, nous avons décidé de prendre les plus récentes versions afin d’avoir le plus d’espèces possibles et de recouvrir au maximum l’arbre de la vie des Opisthocontes et plus précisément des téléostéens pour l’étude 2. 
\par Et afin de valider un moment d’apparition, il pourrait être réalisable de faire des BLAST pour chacun des gènes avec l’espèce ancestrale la plus éloignée d’un point de vue évolutif. 
\par De plus, nous avons été confrontés à de nombreux questionnements concernant la gestion des données KEGG. En effet, KEGG reste une base de données renseignée à la main, et par facilité de compréhension, des voies peuvent afficher une famille de gènes paralogues derrière une étiquette portant un nom plus générique. Seulement, elle ne le fait pas systématiquement et des familles de gènes peuvent se retrouver dans des étiquettes uniques, comme c’est le cas par exemple pour la voie PPAR, les gènes PPAR alpha, beta/delta et gamma sont des paralogues qui sont affichés distinctement dans la voie. L’hypothèse émise était que les paralogues ont des fonctions différentes et des interactions spécifiques comme des interactions communes entre elles, comme semble le montrer la voie. Pour autant, il a fallu décider de ce que l’on faisait dans des cas comme PPAR. Nous nous sommes également demandé s’il n’était pas préférable de gérer tous les paralogues de la même façon et de prendre le même ancêtre commun pour une même famille de gènes. Là encore, nous avons discuté pour savoir quel ancêtre commun nous prenions si les arbres des différents paralogues n’étaient pas d’accord. Prenions-nous le plus ancien, le plus récent, le plus représenté ? Nous avons fini par faire l’étude avec toutes les configurations possibles avant de trancher sur celle qui nous semblait présenter le moins de biais : la gestion des paralogues telle que présenté par KEGG, soit en groupe, soit unique. Et pour les gènes avec de nombreux paralogues en groupe, de prendre le nœud ancestral le plus éloigné. 
\par Par souci de simplification des voies, nous avons également omis l’information de l’interaction, à savoir si l’interaction était activatrice, ou inhibitrice, ou autre. Cependant, il faudrait confronter nos résultats à cette dimension-là. Nous pourrions séparer les interactions en fonction de leur type et regarder si des corrélations se dégagent. 

\newpage
\section{Perspectives}
\par Les résultats que nous avons obtenus concernant le moment d’apparition des gènes peuvent faire l’objet de projets complémentaires. J’aimerais ajuster plus précisément les moments d’apparition en regardant chez les clades plus éloignés encore comme celui des plantes par exemple ou celui des bactéries. Le génome de l’homme comportant environ 25\% de ses gènes en commun avec les plantes, il est fort possible qu’un certain nombre de gènes soient des gènes de voie de signalisation.
\par Par ailleurs, nous aimerions proposer à la base de données KEGG d’implémenter un nouvel outil faisant apparaître les moments d’apparitions des gènes directement sur les voies avec un système de couleur comme nous l’avons réalisé. En poussant nos idées, nous pourrions également afficher le nombre d’orthologues retrouvés pour une espèce donnée, cela permettrait en un coup d’œil de connaître la présence ou non d’un gène chez une espèce, et de connaître la stœchiométrie de ces interactions dans un cadre de voie de signalisation. 
\par Nos résultats concernant la coévolution de gènes impliqués en interactions nous ont donnés envie de regarder ce qu’il en était concernant les gènes pro-apoptotiques et anti-apoptotiques. En effet, il serait intéressant de regarder qui de l’un ou de l’autre apparaît le premier. Cela rejoindrait les études précédentes sur les ligand-récepteur, et les gènes des voies de signalisation. 
\par Bien évidemment, il serait plus juste scientifiquement de valider tous nos résultats avec des approches de biologie expérimentale. Cependant je n’ai pas les outils pour imaginer des expériences poussées, et de plus, en travaillant sur un si large matériel comme nous l’avons fait ici, il faudrait faire des choix. 
