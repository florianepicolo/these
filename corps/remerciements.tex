\chapter*{Remerciements}
\thispagestyle{empty}
\onehalfspacing

\par Cette thèse a été réalisée sous la direction de Philippe Monget au sein de l’équipe Biologie INtégrative des GOnades (BINGO) et sous l’encadrement de Benoît Piégu de l’unité Physiologie de la Reproduction et des Comportements (PRC) à l’INRAE de Nouzilly et grâce aux financements de l’Université de Tours. 
\par Je remercie donc l’ensemble des membres de la direction d’unité et tout particulièrement Sophie Mary, Anne Mychak, Ghyslaine Ploux et Sandra Cavalie pour leur aide administrative.
\par Je tiens principalement à remercier les membres du jury d’avoir accepté d’évaluer mon travail, Julien Bobe et Thomas Faraud en tant que rapporteurs ainsi que Anne Duittoz et Claudine Landes en tant qu’examinatrices. 
\par Un profond merci à mon directeur de thèse Philippe Monget pour avoir cru en moi dès le début de mon premier stage en 2016 puis en 2019 et de m’avoir donné cette chance de réaliser une thèse sur un sujet passionnant. Tu as été un réel soutien sur le plan scientifique bien évidemment, pour tous les conseils durant ces trois années, mais surtout pour ton efficacité sans nom dans les retours que ce soit sur les articles ou ce manuscrit. Et tu as également été une bonne épaule pour moi, la doctorante un peu (\textit{beaucoup}) fragile émotionnellement, et qui n’a pas toujours été facile à gérer. Et je tiens également à te dire que je suis ravie que tu ne viennes pas à ma soutenance avec une perruque bleu-blanc-rouge malgré une défaite douloureuse de la France au Rugby. 
\par Je souhaite également remercier mon encadrant Benoît Piégu, pour le soutien et les déblocages sur R. Merci d’avoir insisté si \textbf{lourdement} pour que je sauvegarde mes avancements sur GIT, tu auras finalement eu raison de moi, malgré ma ténacité à te tenir tête. On se souviendra des petites blagues aux JOBIM 2023, mais aussi de toutes celles durant ces trois ans. Et un merci tout particulier pour ces parties de ping-pong mailistique durant la rédaction.
\par Merci à Elis, le chef de notre équipe BINGO. Je n’aurais finalement jamais goûté aux chocolats que tu prétends donner à tous les membres de l’équipe, venant pour tes superbes conseils dignes d’un psychologue. Merci d’être toi, et de m’avoir intégré dans l’équipe.
\par Merci à tous les membres de l’équipe BINGO : Sébastien Elis, Philippe Monget, Marie- Émilie Lebachelier, Sophie Fouchécourt, Virginie Maillard, Peggy Jarrier-Gaillard, Pascal Papillier, Rozenn Dalbies-Tran, Véronique Cadoret, Sandrine Fréret, Ghylène Goudet, Maria-Teresa Pellicer-Rubio, Catherine Taragnat, Svetlana Uzbekova, Laetitia Corset, Cécile Douet, Stéphanie Martinet, Charline Pontlevoy, Claire Vignault, Ophélie Téteau, Alice Desmarchais et Anna Grandchamp.
\par Un merci tout particulier à Sophie Fouchécourt qui a été la première à croire en moi et à me donner cette chance lors d’un stage en DUT en 2016 alors que je n’étais qu’une novice en bio-informatique. Merci de m’avoir accueilli et de m’avoir fait confiance. 
\par Un grand merci également à Michel Laurin et Jérémie Bardin du MNHN de Paris avec qui nous avons grandement collaboré avec un immense plaisir. Bien que l’on ne se soit jamais rencontré, j’ai adoré travailler avec vous. 
\par Un merci à l’équipe de PIXANIM, Valoch, Ana Paula ou AP, Dany, Mimi, Cholet et Jérôme de m’avoir accueillie dans leur bureau pour prendre le café, se raconter nos péripéties, rigoler, faire des compétitions de pédantix et peindre même à l’occasion ! Chaque instant était un plaisir avec vous. 
\par Je remercie aussi les stagiaires de l’équipe Maoui alias le meilleur stagiaire dont je me souviendrais toujours de l’histoire du vélo dans le bus, et tes skills éclatés au sol sur Excel. Et Mapâte la meilleure piou-piou devenue PIOUUU maintenant avec qui on a beaucoup rigolé de ta drôle de passion pour les diverticules, et d’ailleurs, je reste fan de tes slides animés. 
\par Je vais remercier également les zozos du fond du couloir qui font énormément de bruits sans fermer leur porte, les doctorants de l’équipe SENSOR bien évidemment, dont le Bourdz mon partenaire de crêpe et de chouchen, Loïse celle qui a copié ma date de soutenance, je ne serais pas là pour te voir, mais force ! Et entre autres, Mathy que j’ai allégrement embêté en cette fin de thèse.
\par À Marie et Coline, mes co-thésardes préférées, qui m’ont sorti de l’isolement après 1 an de thèse cloitrée à la maison puis dans un bureau isolé au sous-sol. Marie, tu as été d’un soutien immense durant cette thèse. Merci pour tous tes conseils, mais aussi pour toutes ces conversations dans le bureau, et tous ces BeReal ensemble. Oui, c’est Michel, tu donnes pas de nouvelles. Et Coline la meilleure thésarde du monde, un modèle d’exemple d’exemplarité ! Sous tes petits mots fleuris, j’ai su déceler tout l’amour que tu m’apportais, et finalement… bon, je t’aime aussi un peu en retour. Avec toi, c’est l’amour vache (je suis hyper drôle) mais j’ai passé que des bons moments à gueuler avec toi (\textit{et sur toi}). D’ailleurs, je t’offre ma voiture, parce que finalement tu l’as plus conduite que moi, et on sait pourquoi. Finalement, à linera, on ne se fait pas que des collègues, on se fait aussi des bêtes de copines.
\par Un énorme merci aussi à Mimi, la plus fun de ma vie, qui prend si soin de moi. Tu m’auras mise une musique par jour dans la tête, tu m’auras épuisée de ton trop pleins d’énergie et de tous tes uppercuts, mais tu vas me manquer quand tu seras de l’autre côté de l’Atlantique. J’ai adoré te faire des chasses aux trésors et dessiner sur ton bureau pour voir mon bureau multicolore en retour. C’était de bonne guerre. Et un big up à Pouf, Joël et sa poire.
\par Et puis 3 ans, c’est aussi partager son quotidien avec ses voisins ! Merci aux Velpotes : Romain, Pierre, Valentin, Léa et Mathilde. Vous avez rythmé ma vie en partageant des soirées jeux, des pétanques ou des dîners, et vous m’avez même remotivé à courir ! On a vécu des moments improbables ensemble comme l’étrangleur ou Bertrand Desplantes qui ont rempli ma boite à souvenirs. Et un petit mot pour Valentin, mon copain de DUT, merci d’avoir repris contact, merci à nos soirées télés, et de m’avoir fait rencontrer Doriane, cette merveilleuse personne que je vois maintenant plus que toi finalement… Vous êtes au top les velpotes. 
\par J’ai également rencontré de formidables personnes grâce à l’Association des Doctorants de Tours dont j’ai été membre pendant mes 3 années et grâce à qui j’ai rencontré mes copaines de Toursnez Ménage qui m’ont chaleureusement accueilli dans leurs bureaux : Antonin, Yopo, Léa, Thomas, Romain, Igor, Yegor, Guillaume, Théo, Adam, Pierre, Sylvain, Merve, Maxime et Marion ! Que des numéros un dans cette team ! Ça aura été un plaisir de vous gronder chaque jeudi pour que vous veniez faire la fête ! 
\par Dans cette même veine, j’aimerais faire un aparté pour Romain, mon voisin et maintenant meilleur ami, j’ai passé de merveilleux moments avec toi, à rigoler et faire les andouilles (\textit{qu’on adore}). Puis notre trio de l’enfer avec Thomas, à chanter sur les quais de Loire ou juste à se raconter nos vies sur la place Velpeau jusqu’à pas d’heure. Et aussi merci de m’avoir fait rencontrer Sam, notre artiste qui m’a aidé à décompresser pendant la rédaction avec nos jeux de coop. Merci !
\par Et puis je suis obligé de remercier dignement mes parents adoptifs du Shamrock, Annie et Fafa ! Deux formidables personnes, à qui je souhaite un bon départ vers de nouveaux horizons. Annie, merci d’avoir été ma maman à Tours, toujours à nous vouloir nous faire plaisir et à prendre soin de nous. Et que vive longtemps le PICOLO Grande !
\par J’aimerais également remercier du fond de mon cœur, le serveur discord des doctorants « PhD Student » que j’ai rejoint le premier jour de thèse, et que je n’ai jamais lâché depuis. De nombreuses amitiés se sont créées grâce à cette plateforme en ligne, et on a pu se soutenir et rire pendant la pandémie, mais bien plus après notamment à travers les IRL, mais aussi en comptant l’heure (\textit{ça n’a pas de sens}) mais merci aux horlogistes. 
\par Un grand merci à Anana, ma plus belle rencontre sur ce serveur, qui au premier abord n’était qu’une camarade de thèse, mais qui s’est transformé en amitié si intense. Je ne compte plus les allers-retours jusqu’à Lille pour passer un week-end avec toi et les dinos LoloR et Ouistiti. Merci d’être la personne que tu es, tu m’as fait grandir sur bien des sujets et notamment la communication. Finalement, si j’arrive à m’ouvrir dans ses remerciements, c’est peut-être grâce à toi. 
\par Un merci à Galaad aussi, j’ai adoré nos échanges, nos tierlists, et nos jeux (\textit{achetés pour ne plus y toucher}), notre projet de tout plaquer pour lancer une chaîne Twitch n’aura finalement pas eu lieu, mais on avait un si beau concept. Je le garde en tête. Et un gros signe JuL à notre groupe Salut l’Ariège, et à nos AG exceptionnelles pour organiser nos vacances. 
\par Un merci plus général aux équipes de modération du serveur « PhD Student ». En deux ans de modération, j’en ai vu des camarades, et j’ai adoré travailler avec chacun d’entre vous. Un jour, je partirais, mais je suis sûre que mes poulains sauront rédiger des comptes rendus qui me rendront digne (\textit{j’en fais des tonnes}). Courge ! 
\par Et puis, jamais très loin, j’ai les copains du master, on ne s’est pas vu normalement, mais c’était pour profiter encore plus lorsque l’on se retrouvait aux BIGDAY, aux JOBIM ou lors des soutenances de thèses. À ma meilleure amie d'étude supérieur, Manon, qui est maintenant Docteure. Tu es la meilleure ne l’oublie pas. 
\par Je remercierai également mes amies du lycée et principalement Angèle et Laurie, mes deux meilleures amies depuis si longtemps et avec qui on ne s’est jamais lâché. Chaque instant avec vous vaut le dé\textit{Tours} (jeu de mot, parce que je reste l’humour du groupe). J’espère qu’un jour les gars (\textit{et nous}) finiront par grandir et on pourra partir sereinement au ski. Enfin, quoique, ça fait tellement de bons souvenirs. 
\par Et enfin ma famille, je ne sais si je trouverai les mots pour vous remercier. À mon père et ma belle-mère qui m’ont toujours écouté me plaindre sans forcément comprendre. À ma maman parfaite et Sergio qui pensent encore que je vais bientôt devenir Avocate, désolée c'est toujours pas ça. À mon frère à qui je souhaite de réussir, tu as des mains en or ma breur. À mon arrière-grand-mère décédée, et à mon arrière-grand-père qui nous a également quitté durant ma rédaction. À mon grand-père et à ma merveilleuse grand-mère. Ça n’a jamais été facile de se dire je t’aime, mais je crois que je peux au moins l’écrire. \textbf{Je vous aime.} 
\par Et puis un petit merci, à mon chat, Nuggets. Ce gros sac à prout a rythmé mes nuits et mes insomnies. Un enfant capricieux, mais que l’on aime de tout son cœur. 
\par Alors voilà que s’achève 3 années de ma vie. Quand je repense à Philippe le 15 novembre 2020 me disant « tu verras, 36 mois ça passe très vite » et moi lui riant au nez, finalement, tu avais raison. En 36 mois, j’ai eu trois covids compliqués, dont un durant ma soutenance, une grippe, un accident de voiture, et je suis aussi passée sous une voiture. J’ai eu le feu dans mon appartement et deux dégâts des eaux. Je me suis fait voler mes affaires sur les bords de Loire et j’ai eu un étrangleur dans mon immeuble. Mais j’ai également j’ai fait partie de deux associations incroyables, créé un jeu de piste sur la ville de Lyon et créé un cocktail au Shamrock de Tours, organisé deux IRL géantes pour les doctorant.es en France, couru 10 km et battu mon propre record. 
\par\textbf{\textit{Mais surtout, j’ai réussi à écrire cette thèse.}}  

% \pagestyle{empty}
% \AddToShipoutPictureBG*{%
%   \AtPageLowerLeft{%
%     \includegraphics[width=0.2\paperwidth,height=0.2\paperheight]{figures/logos/nugg.png}%
%   }%
% }

\begin{flushright}
Merci à vous.\\
Bravo à moi.
\end{flushright}
% \pagestyle{empty}
\newpage